%  LaTeX support: latex@mdpi.com 
%  In case you need support, please attach all files that are necessary for compiling as well as the log file, and specify the details of your LaTeX setup (which operating system and LaTeX version / tools you are using).

%=================================================================
% \documentclass[ijgi,submit,moreauthors,pdftex]{Definitions/mdpi} 
\documentclass[preprints,ijgi,submit,moreauthors]{Definitions/mdpi} 
% If you would like to post an early version of this manuscript as a preprint, you may use preprint as the ojurnal and change 'submit' to 'accept'. The document class line would be, e.g., \documentclass[preprints,article,accept,moreauthors,pdftex]{mdpi}. This is especially recommended for submission to arXiv, where line numbers should be removed before posting. For preprints.org, the editorial staff will make this change immediately prior to posting.
\usepackage{textcomp}
\usepackage{gensymb}
\usepackage{subcaption}

%---------
% article
%---------
% The default type of manuscript is "article", but can be replaced by: 
% abstract, addendum, article, benchmark, book, bookreview, briefreport, casereport, changes, comment, commentary, communication, conceptpaper, conferenceproceedings, correction, conferencereport, expressionofconcern, extendedabstract, meetingreport, creative, datadescriptor, discussion, editorial, essay, erratum, hypothesis, interestingimages, letter, meetingreport, newbookreceived, obituary, opinion, projectreport, reply, retraction, review, perspective, protocol, shortnote, supfile, technicalnote, viewpoint
% supfile = supplementary materials

%----------
% submit
%----------
% The class option "submit" will be changed to "accept" by the Editorial Office when the paper is accepted. This will only make changes to the frontpage (e.g., the logo of the journal will get visible), the headings, and the copyright information. Also, line numbering will be removed. Journal info and pagination for accepted papers will also be assigned by the Editorial Office.

%---------
% pdftex
%---------
% The option pdftex is for use with pdfLaTeX. If eps figures are used, remove the option pdftex and use LaTeX and dvi2pdf.

%=================================================================
\firstpage{1} 
\makeatletter 
\setcounter{page}{\@firstpage} 
\makeatother
\pubvolume{xx}
\issuenum{1}
\articlenumber{5}
\pubyear{2020}
\copyrightyear{2020}
%\externaleditor{Academic Editor: name}
\history{Received: date; Accepted: date; Published: date}
%\updates{yes} % If there is an update available, un-comment this line

%% MDPI internal command: uncomment if new journal that already uses continuous page numbers 
%\continuouspages{yes}

%------------------------------------------------------------------
% The following line should be uncommented if the LaTeX file is uploaded to arXiv.org
%\pdfoutput=1

%=================================================================
% Add packages and commands here. The following packages are loaded in our class file: fontenc, calc, indentfirst, fancyhdr, graphicx, lastpage, ifthen, lineno, float, amsmath, setspace, enumitem, mathpazo, booktabs, titlesec, etoolbox, amsthm, hyphenat, natbib, hyperref, footmisc, geometry, caption, url, mdframed, tabto, soul, multirow, microtype, tikz

%=================================================================
%% Please use the following mathematics environments: Theorem, Lemma, Corollary, Proposition, Characterization, Property, Problem, Example, ExamplesandDefinitions, Hypothesis, Remark, Definition, Notation, Assumption
%% For proofs, please use the proof environment (the amsthm package is loaded by the MDPI class).

%=================================================================
\Title{Spatiotemporal Analysis of Bike Sharing System during the COVID-19 Epidemics: a Case Study of Beijing}

\newcommand{\orcidauthorA}{0000-0001-7546-7068} % Add \orcidA{} behind the author's name
\newcommand{\orcidauthorB}{0000-0003-0084-381X} % Add \orcidB{} behind the author's name

\Author{Xinwei Chai $^{1,2,\ddagger}$\orcidA{}, Xian Guo $^{1,\dagger,\ddagger,*}$\orcidB{} and Jie Jiang $^{1}$}

\AuthorNames{Xian guo, Xinwei Chai and Jie Jiang}

\address{%
$^{1}$ \quad Beijing University of Civil Engineering and Architecture, 102616 Beijing, China; guoxian@bucea.edu.cn (X.G.); jiangjie@bucea.edu.cn (J.J.)\\
$^{2}$ \quad China Location-Based Service, 100191 Beijing, China; xw.chai@chinalbs.org}

\corres{Correspondence:  guoxian@bucea.edu.cn; Tel.: +86-010-6120 9335}

\abstract{During the epidemics of COVID-19, the whole world is experiencing a huge crisis on public health and economy.
As estimating the overall loss is troublesome, we try to reveal the aspect of behavior changes of Beijing citizens in order to infer the impact on people's daily life and economics.
Big data of BSS (Bike Sharing System) ...
Method and conclusion ...
}

\keyword{Bike sharing system; coronavirus; spatiotemporal analysis}


%%%%%%%%%%%%%%%%%%%%%%%%%%%%%%%%%%%%%%%%%%
% Only for the journal Data:
%\dataset{DOI number or link to the deposited data set in cases where the data set is published or set to be published separately. If the data set is submitted and will be published as a supplement to this paper in the journal Data, this field will be filled by the editors of the journal. In this case, please make sure to submit the data set as a supplement when entering your manuscript into our manuscript editorial system.}

%\datasetlicense{license under which the data set is made available (CC0, CC-BY, CC-BY-SA, CC-BY-NC, etc.)}
\Urlmuskip=0mu plus 10mu
\begin{document}
\section{Introduction}
In December 2019, a local outbreak of pneumonia was detected in Wuhan, China, identified as COVID-19 and the virus as SARS-CoV-2\footnote{\url{https://www.who.int/emergencies/diseases/novel-coronavirus-2019/technical-guidance/naming-the-coronavirus-disease-(covid-2019)-and-the-virus-that-causes-it}}.
COVID-19 was quickly spread to the whole country and the whole world.
According to a situation report of WHO\footnote{\url{https://www.who.int/docs/default-source/coronaviruse/situation-reports/20200329-sitrep-69-covid-19.pdf}}, the total confirmed cases have reached 634,835 as of 29 March, 2020.

COVID-19-related studies have been mostly conducted on epidemiology based on the timeline of outbreak, e.g. transmission dynamics \cite{li2020early} and preventive measures \cite{chinazzi2020effect}.

It is well-accepted that COVID-19 inflict huge impact on public health and all the domains of economy, from spring festival till today (March 2020).
However, few studies have been conducted to analyze quantitatively the behavior pattern of people before and during the epidemics.
The wide-spread Bike Sharing System (BSS)  in China offers a possibility for analyzing such pattern of residents.

The 3rd generation of BSS emerges in China in 2015 thanks to rapid development of GIS-based and IoT-based system. 
In this BSS, bikes are no longer constrained by docking stations (so-called free floating), they often spread along the roads, congest around shopping/residential areas, which cover the most urban residents. 
As for city of Beijing, the amount of share bike reached its peak in 2017 and governors began removing excess supply afterwards. 
The demand and the supply of share bikes meet a balance in 2019, which could be a stable data source.
According to Daxue Consulting \cite{bssmodel}, in Beijing, 93\% of travels less than 5km are quicker done by bike and public transport than with the car, which suggests share bikes has a potential reflection on the local mobility of residents.

Xu et al. characterizes the BSS spatiotemporal flow and in Singapore \cite{xu2019unravel}.
Kaggle competition of predicting share bike demands \cite{kaggle}.
There are also studies of 3rd generation BSS focusing on rebalancing strategies \cite{pal2017free, ai2019deep}.

Our work aims at analyzing the differences between the patterns of epidemics and that of ordinary days in order to estimate the impact of the epidemics.

\textbf{Contribution:} people's behavior pattern, share bike usage $\to$ impact on economy.

\section{Materials and Methods}

\begin{figure}[H]
    \centering
    \includegraphics[width=\textwidth]{Definitions/StudyArea.png}
    \caption{Location map and 16 administrative districts of Beijing}
    \label{fig:study_area}
\end{figure}

\subsection{Study area}
This study is conducted in the city of Beijing, the capital of China. 
As shown in Figure \ref{fig:study_area}, Beijing locates at the North China Plain, occupying an area of 16,411km$^2$ ( 39.4{\degree}-41.6{\degree}N, 115.7{\degree}-117.4{\degree}E). 
In 2019, the municipal population of Beijing reached 21.53 million. According to Beijing Health Commission, 352 confirmed cases\footnote{\url{http://wjw.beijing.gov.cn/xwzx_20031/wnxw/202002/t20200212_1628835.html} (in Chinese)} were found in Beijing by February 10, 2020. 
As an imported metropolis, Beijing has taken measures in response to the outbreak. 
Under this circumstance, the activities of residents have been influenced and showing special spatiotemporal patterns.

\subsection{Data Sets}
(a) \textbf{BSS spatiotemporal data} Temporal positioning datasets come from 900 thousand share bikes belonging to 3 main BSS operators (Mobike, DiDi Bike, Hellobike and Ofo) in Beijing.
The datasets date from March 2018 to March 2020 (111.7 GB) and cover 1.5 million usages per day contributed by 11 million users which account for one half of the total population of Beijing.
Records contain the positioning and timing information of locking \& unlocking bikes, excluding that of rebalancing operations.
Certain districts (Chaoyang, Fengtai and Shijingshan) are not comprised in the datasets due to different policies of local governments.

(b) \textbf{The infected residential areas} was collected from the daily update on the novel coronavirus pneumonia outbreak provided by Beijing Health Commission. 
This data records the confirmed cases of each district from Jan 20 to Mar 5, 2020. 

{\centering\textit{A sample table should be presented here}

}

\textbf{[Guo]}(c) \textbf{Points of interests (POIs) data} was collected from AutoNavi API\footnote{\url{https://lbs.amap.com/api/webservice/guide/api/georegeo} (in Chinese)} and classified into seven categories as shown in table x. 

{\centering\textit{A sample table should be presented here}

}

From the timeline \cite{li2020early} of the outbreak, 

\subsection{Methods}
We proceeded the spatiotemporal data with GeoSpark SQL \cite{huang2017geospark}, which performs spatial queries via parallel computing.
spatial join 
heat map

Basic statistics on the evolution over time in different zones of Beijing. 

\section{Result and Analysis}
\subsection{Temporal patterns of cycling activities}
To characterize the behavior patterns, we select the following important referential dates: 

\begin{table}[H]
    \centering
    \begin{tabular}{ll}
    04 Feb 2019 & Start of Chinese Lunar New Year holiday 2019 \\
    10 Feb 2019 & End of new year holiday 2019\\
    07 Jan 2020 & Identification of COVID-19\\
    22 Jan 2020 & Shut down of Wuhan and other 15 cities\\
    24 Jan 2020 & Start of Chinese Lunar New Year holiday 2020\\
    02 Feb 2020 & End of extended New Year holiday 2020\\
    10 Feb 2020 & Partial restart of social activities
    \end{tabular}
    \caption{Referential dates}
    \label{tab:my_label}
\end{table}


\begin{figure}[H]
    \centering
    \includegraphics[width=\textwidth]{Compare_temporal.eps}
    \caption{Comparison between share bike daily usages of year 2019 and 2020.}
    \label{fig:temporal_comparison}
\end{figure}

We compared share bike daily usage of year 2019 and 2020 from Jan 01 to early March after removing the data of Feb 29, 2020 (leap year).
As can be seen in Figure \ref{fig:temporal_comparison}, in 2019, the usage drops down to $5\times 10^5$ because of spring festival; 
while in 2020, the usage decreases even two thirds of that in 2019 due to the outbreak of COVID-19 and the following epidemics.

Time-series analysis: whole series and statistics of series by weekdays.
Figures and analysis for (a)The hourly number of trips across 2019-2020.
(b)The hourly number of different periods (before Chinese New Year, during the quarantine period, epidemics mitigated). 

\subsection{Spatiotemporal patterns of cycling activities\textbf{[Guo]}}

\begin{figure}[H]
    \begin{subfigure}{.3\textwidth}
        \includegraphics[width=\textwidth]{Definitions/D2020_01_09.png}
        \caption{Before the closure}
    \end{subfigure}
    \begin{subfigure}{.3\textwidth}
        \includegraphics[width=\textwidth]{Definitions/D2020_02_05.png}
        \caption{During the quarantine}
    \end{subfigure}
    \begin{subfigure}{.3\textwidth}
        \includegraphics[width=\textwidth]{Definitions/D2020_02_24.png}
        \caption{Partial restart}
    \end{subfigure}
    \centering
    \caption{Caption}
    \label{fig:my_label}
\end{figure}

Overall visualization of different periods (before Chinese New Year, during the quarantine period, epidemics mitigated)

By removing the factor of Chinese new year, we observe a steep fall of overall share bike usage from 22 January to 29 Feb.
Industrial: IT,...
Residential: ...


\section{Discussion}
\subsection{Correlation analysis between cycling activities and infected communities\textbf{[Guo]}}

\section{Conclusion}


%%%%%%%%%%%%%%%%%%%%%%%%%%%%%%%%%%%%%%%%%%
\vspace{6pt} 

%%%%%%%%%%%%%%%%%%%%%%%%%%%%%%%%%%%%%%%%%%
%% optional
%\supplementary{The following are available online at \linksupplementary{s1}, Figure S1: title, Table S1: title, Video S1: title.}

%%%%%%%%%%%%%%%%%%%%%%%%%%%%%%%%%%%%%%%%%%
%% optional
\abbreviations{The following abbreviations are used in this manuscript:

\noindent 
\begin{tabular}{@{}ll}
GIS & Geographic Information System \\
BSS & Bike Sharing System

\end{tabular}}

%%%%%%%%%%%%%%%%%%%%%%%%%%%%%%%%%%%%%%%%%%
%% optional
\appendixtitles{no} %Leave argument "no" if all appendix headings stay EMPTY (then no dot is printed after "Appendix A"). If the appendix sections contain a heading then change the argument to "yes".
\appendix
\section{}
\unskip
\subsection{}

%%%%%%%%%%%%%%%%%%%%%%%%%%%%%%%%%%%%%%%%%%

\reftitle{References}

\externalbibliography{yes}
\bibliography{bib}

%%%%%%%%%%%%%%%%%%%%%%%%%%%%%%%%%%%%%%%%%%
%% optional
% \sampleavailability{Samples of the compounds ...... are available from the authors.}

%% for journal Sci
%\reviewreports{\\
%Reviewer 1 comments and authors’ response\\
%Reviewer 2 comments and authors’ response\\
%Reviewer 3 comments and authors’ response
%}

%%%%%%%%%%%%%%%%%%%%%%%%%%%%%%%%%%%%%%%%%%
\end{document}

